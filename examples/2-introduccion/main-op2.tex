\documentclass{IEEEtran}
\usepackage[spanish]{babel}
\usepackage[utf8]{inputenc}
\usepackage{amsmath}
\usepackage{blindtext}
\usepackage{graphicx}
\title{Curso de Redacción de Artículos Técnico-Científicos con \LaTeX en el ITVM}
\author{Gerardo Marx Chávez-Campos}
\begin{document}
\maketitle
\section{Introducción}
\blindtext[1]

Quiero hacer una cita bibliográfica \cite{einstein}. Algunas otras
citas se pueden ver en \cite[pag.25]{knuthwebsite,latexcompanion}.
%\bibliographystyle{ieeetr}
%\bibliography{referencias}
%-------------------
\begin{thebibliography}{9}
\bibitem{latexcompanion} 
Michel Goossens, Frank Mittelbach, and Alexander Samarin. 
\textit{The \LaTeX\ Companion}. 
Addison-Wesley, Reading, Massachusetts, 1993.
 
\bibitem{einstein} 
Albert Einstein. 
\textit{Zur Elektrodynamik bewegter K{\"o}rper}. (German) 
[\textit{On the electrodynamics of moving bodies}]. 
Annalen der Physik, 322(10):891–921, 1905.
 
\bibitem{knuthwebsite} 
Knuth: Computers and Typesetting,
\\\texttt{http://www-cs-faculty.stanford.edu/index.html}
\end{thebibliography}
\end{document}