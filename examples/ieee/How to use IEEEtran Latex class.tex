\documentclass[9pt]{IEEEtran}
\usepackage{amsmath}
\usepackage{graphicx}
\usepackage{lipsum}
\usepackage[utf8]{inputenc}
%\usepackage[spanish]{babel}
\title{How to Use the IEEEtran \LaTeX{} Class}
\author{Chávez-Campos~Gerardo~Marx,
~\IEEEmembership{Member,~IEEE.}%
\thanks{This work was supported by the IEEE.}}
\begin{document}
\maketitle

\begin{abstract}
Put your abstract here...
\end{abstract}


\begin{IEEEkeywords}
Broad band networks, quality of service, WDM.
\end{IEEEkeywords}

\section{Introduction}
\IEEEPARstart{T}{he} LED

\begin{table}[!t]
\renewcommand{\arraystretch}{1.3}
\caption{A Simple Example Table}
\label{table_example}
\centering
\begin{tabular}{cc}
\bfseries First & \bfseries Next\\
\hline
1.0 & 2.0\\
\hline
\end{tabular}
\end{table}

Como lo indica el autor \cite{gratzer2013math}


\bibliography{referencias}
\bibliographystyle{IEEEtran}

\end{document}